
\documentclass[a4paper,12pt]{article}
\usepackage{times}
\usepackage[francais]{babel}
\usepackage[utf8x]{inputenc}
\usepackage[T1]{fontenc}
\usepackage{amsmath}
\usepackage{amssymb}
\usepackage{graphicx}
\usepackage{pdfpages}
\usepackage{pdflscape}
\usepackage{listings}
\usepackage{longtable}
\lstset{literate=
{é}{{\'e}}1
{è}{{\`e}}1
{ê}{{\^e}}1
{à}{{\`a}}1
{â}{{\^a}}1
}
\lstset{language=C++,
basicstyle=\footnotesize,
keywordstyle=\footnotesize\color{blue},
otherkeywords={override,nullptr}
}
\definecolor{orange}{rgb}{0.8,0.4,0.0}
\definecolor{darkblue}{rgb}{0.0,0.0,0.6}
\definecolor{cyan}{rgb}{0.0,0.6,0.6}
\lstdefinelanguage{JSON}
{
basicstyle=\normalsize,
columns=fullflexible,
showstringspaces=false,
commentstyle=\color{gray}\upshape,
morestring=[b]",
morestring=[s]{>}{<},
morecomment=[s]{<?}{?>},
stringstyle=\color{orange},
identifierstyle=\color{darkblue},
keywordstyle=\color{blue},
morekeywords={string,number,array,object}% list your attributes here
}

\sloppy

\setlength{\topmargin}{0cm}
\setlength{\headsep}{0.in}
\setlength{\headheight}{0.in}
\setlength{\evensidemargin}{0cm}
\setlength{\oddsidemargin}{-1cm}
\textwidth 18cm
\textheight 25cm

\begin{document}

    \thispagestyle{empty}

    \begin{titlepage}

        \vspace*{2cm}

        \begin{center}\textbf{\Huge Projet Logiciel Transversal}\end{center}{\Large \par}

        \begin{center}\textbf{\large Anand Candassamy \& Paul Estano}\end{center}{\large \par}

        \vspace{2cm}

        \clearpage

        {\small
        \tableofcontents
        }

    \end{titlepage}

    \clearpage
    \section{Présentation Générale}

    \subsection{Archétype}
	    Notre jeu s'inspirera principalement du jeu \emph{Pokemon Donjon Mystère}. En effet, nous avons prévu de conserver le mécanisme des combats et de donjon de ce jeu.
	    \\Dans notre logiciel l'utilisateur incarnera un pokemon dans les salles d'un donjon qui contiennent chacune des pokemons qui peuvent l'"agresser".
	    \\Pour simplifier le jeu nous abandonnons également d'évolution des pokemons.

    \subsection{Règles du jeu}
	  Le donjon contient un nombre fini de salle et le joueur gagne lorsqu'il sort de la dernière salle du donjon.
	  Un donjon a 10 étages, le pattern de chaque étage n'est pas tiré aléatoirement. Le joueur a la possibilité d'accéder à des étages bonus, dîtes arène de combat où il pourra se battre contre un autre joueur en ligne.
	   \\Le joueur est provoqué en duel automatiquement par les pokemons qui sont autour de lui.
\\Les combats fonctionnent en tour par tour. A chaque tour, l'utilisateur peut effectuer qu’une seule action :
\begin{itemize}
    \item attaquer
    \item soigner le pokemon
    \item se déplacer
\end{itemize}
\clearpage
    \subsection{Ressources}
    \begin{figure}[ht]
    \begin{center}
        \includegraphics[width=0.8\textwidth]{tilemap.png}
    \end{center}
    \caption{Tileset utilisé pour construire un monde}
    \end{figure}
    
    \begin{figure}[ht]
    \begin{center}
        \includegraphics[width=0.8\textwidth]{pokemonLatex.png}
    \end{center}
    \caption{Tileset utilisé pour les pokémons}
    \end{figure}



    \clearpage
    \section{Description et conception des états}
    
    \subsection{Description des états}
    Un état du jeu est formé d'une carte qui est statique au cours du temps et d'une liste de joueurs dont l'état varie au cours du jeu.
    
     \subsubsection{État de la carte}
     
     La carte est un élément fixe de l'état du jeu. Il se décompose en plusieurs cases.
    
     Les cases se distinguent en 3 parties : \begin{itemize}
    \item les murs, qui ne sont pas franchissables par les éléments mobiles.
    \item les cases "conteneurs", elles peuvent être vide ou occupé par un seul élément mobile.
    \item les cases escaliers, il en existe un seul par étage, qui permet d'aller au niveau suivant du labyrinthe.
    
    \end{itemize}

    \subsubsection{État des joueurs}
    Un pokemon est soit contrôlé par un joueur soit contrôlé par une Intelligence Artificielle (I.A), il possède une position, un nombre de points de vie à l'état t, un nom et un ensemble de statistiques qui lui sont attribués en début de partie. Le pokemon est considéré comme étant "mort" si son nombre de points de vie atteint 0.
    
    \subsubsection{État général}
    A l’ensemble des éléments statiques et mobiles, nous rajoutons les propriétés suivantes :
    — Époque : représente « l’heure » correspondant à l’état, ie c’est le nombre de « tour » passés
    globale depuis le depuis de la partie.

    
    Ces statistiques correspondent aux attaques qu'il peut utiliser, à son nombre de point de vie en début de partie et son type (eau, feu, herbe).
    \subsection{Conception Logiciel}
    Le package état peut se diviser en trois sous-partie:\begin{itemize}
        \item Une partie gérant les personnages, en bleu
        \item Une partie gérant l'environnement, en rouge
        \item Une classe représentant l'état global du jeu, en vert
    \end{itemize}
    
     La classe \emph{player} contient l'ensemble des éléments permettant de caractériser l'état d'un joueur. Chaque joueur est lié à un pokemon par une relation de composition : un pokemon ne peut pas exister sans joueur. Dans le cas où le pokemon est contrôlé par l'IA, l'IA est considérée comme un joueur et contrôle donc son pokemon.
    
    Chaque pokemon possède une relation d'agrégation avec des statistiques qui lui sont attribués en début de partie.
    
    La partie environnement est calqué sur le modèle proposé par les fichiers exporter par le logiciel \emph{Tiled Map Editor}. Les objets les plus consommateurs en mémoire tels les \emph{data} dans la classe \emph{Layer} sont par ailleurs stockés dans le tas pour éviter les copies entre les différentes classes.
    
     L'état global contient un pointeur vers l'état de l'environnement et et une liste de pointeur de contenant l'état des différents joueurs de la partie.
    \begin{landscape}
    \begin{figure}[p]
    \includegraphics[width=0.8\paperheight]{state.png}
    \caption{\label{uml:state}Diagramme des classes d'état.}
    \end{figure}
    \end{landscape}
    \clearpage
    
    
    \section{Rendu: Stratégie et Conception}

    \subsection{Stratégie de rendu d'un état}


    \subsection{Conception logiciel}

    %\begin{landscape}
    %\begin{figure}[p]
    %\includegraphics[width=0.9\paperheight]{render.pdf}
    %\caption{\label{uml:render}Diagramme des classes de rendu.}
    %\end{figure}
    %\end{landscape}

    \clearpage
    \section{Règles de changement d'états et moteur de jeu}

    \subsection{Règles}

    \clearpage
    \subsection{Conception logiciel}


    %\begin{landscape}
    %\begin{figure}[p]
    %\includegraphics[width=0.9\paperheight]{engine.pdf}
    %\caption{\label{uml:engine}Diagramme des classes de moteur de jeu.}
    %\end{figure}
    %\end{landscape}


    \section{Intelligence Artificielle}

    \subsection{Stratégies}

    \clearpage
    \subsection{Conception logiciel}


    %\begin{landscape}
    %\begin{figure}[p]
    %\includegraphics[width=0.9\paperheight]{ai.pdf}
    %\caption{\label{uml:ai}Diagramme des classes d'intelligence artificielle.}
    %\end{figure}
    %\end{landscape}


    \section{Modularisation}
    \label{sec:module}

    \subsection{Organisation des modules}

    \clearpage
    \subsection{Conception logiciel}
    
    \clearpage
    \section{Annexe}
    \label{sec:Annexe}
    \subsection{Bibliographie}

    %
    %\begin{landscape}
    %\begin{figure}[p]
    %\includegraphics[width=0.9\paperheight]{module.pdf}
    %\caption{\label{uml:module}Diagramme des classes pour la modularisation.}
    %\end{figure}
    %\end{landscape}

\end{document}
